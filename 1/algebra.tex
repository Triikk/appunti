\documentclass{article}
\usepackage{graphicx} % Required for inserting images
\usepackage{amsmath}
\usepackage{mathtools}
\usepackage{amsfonts} % \mathbb
% \usepackage{centernot}

\title{
	Appunti di Algebra \\
	\large Corso di Ingegneria e Scienze Informatiche - 1 anno
}
\author{Mattia Ronchi}
\date{}

\begin{document}

\maketitle

\section{Introduzione}

\subsection{Relazioni}
Una \textbf{relazione} é un sottoinsieme del prodotto cartesiano di due o piú insiemi.\\
Una relazione su $A$ é un sottoinsieme di $A \times A$. \\\\
$a_{1}$ é in relazione con $a_{2}$ e si scrive $a_{1} R a_{2}$. \\
Def. Una relazione é di \textbf{equivalenza} se rispetta le seguenti proprietá:
\begin{itemize}
	\item[] Riflessiva: $a R a$ $\forall a \in A$ (ogni elemento é in relazione con se stesso)
	\item[] Simmetrica: $a_{1} R a_{2} \implies a_{2} R a_{1}$ $\forall a_{1}, a_{2} \in A$
	\item[] Transitiva: $a_1 R a_2 \wedge a_2 R a_3 \implies a_1 R a_3$
\end{itemize}

\subsection{Funzioni/Applicazioni}
$f : X \rightarrow Y$\\\\
$f$ iniettiva: $\forall x_1, x_2 \in X, f(x_1) = f(x_2) \implies x_1=x_2$ \\
$f$ suriettiva: $\forall y \in Y, \exists x \in A : y = f(x)$ \\
$f$ biettiva: $\forall y \in Y, \exists! x \in A : y = f(x)$

\subsection{Insiemi numerici}
L'insieme dei numeri razionali $\mathbb{Q}$ introduce gli inversi del prodotto (es. $3 \rightarrow \frac{1}{3}$). \\
L'insieme dei numeri reali $\mathbb{R}$ introduce limiti, radici e altri valori. \\
L'insieme dei numeri complessi $\mathbb{C}$ introduce le radici di indice pari di numeri negativi tramite l'unitá immaginaria $i$ e i suoi multipli. Un numero complesso é esprimibile in forma polare come $a+ib$, con $a,b \in R$.

\subsection{Campi}
$(K, +, \cdot)$ é un campo se:
\begin{itemize}
	\item[] $+, \cdot$ sono associative ($a+(b+c) = (a+b)+c$), commutative ($a+b=b+a$) e distributive ($a(b+c)=ab+ac$)
	\item[] esistono elementi \textbf{neutri} ($0$ per la somma ($a+0=a$), $1$ per il prodotto ($a \cdot 1 = a$)) e \textbf{opposti} ($-a$ per la somma ($a-a=0$), $x^{-1}$ per il prodotto ($x \cdot x^{-1} = 1$), che restituscono il valore neutro
\end{itemize}
Alcuni insiemi campi sono $\mathbb{Q}, \mathbb{R}, \mathbb{C}$.

\subsubsection{Campi finiti}
Dato un numero intero $n \ge 0$, definiamo su $\mathbb{Z}$ la relazione di equivalenza
$$a \equiv b (n) \iff \exists k \in Z : a - b = k \cdot n$$
essa rispetta tutte e 3 le proprietá elencate sopra. \\\\
Definiamo $[b] = \{a \in Z : a \equiv b (n)\}$ e $Z_n = \{[0], [1], ..., [n-1]\}$. \\
Es. in $Z_2 = \{[0], [1]\}$, $[0]$ sono i numeri pari, $[1]$ quelli dispari. \\\\
Definiamo su $Z_n$ le operazioni:\\
$$[a] + [b] = [a+b], [a] \cdot [b] = [a \cdot b]$$ \\
Es. Possiamo scrivere, con la notazione dei campi finiti, il prodotto tra numeri interi: \\
Dato $Z_2$: $[0] \cdot [0] = [0], [0] \cdot [1] = [0 \cdot 1] = [0], [1] \cdot [1] = [1 \cdot 1] = [1]$. \\\\
$Z_n$ é un campo $\iff n$ é \textbf{primo}. Se $n$ non é primo, non esisterá l'inverso di un fattore di $n$, ovvero non esisterá nessuna classe di elementi che se moltiplicata con la classe del fattore restituisca classe 1.

\subsection{Spazi vettoriali}
Uno \textbf{spazio vettoriale} definito su un campo $K$ é un insieme $V$ con due operazioni:
\begin{itemize}
	\item[] $+ : V \times V \rightarrow V$ ($v_1, v_2) \rightarrow v_1 + v_2$)
	\item[] $\cdot : K \times V \rightarrow V$ ($a, v \rightarrow av$)
\end{itemize}
che verificano le seguenti proprietá: $+$ é commutativa, associativa, con elem. neutri (vettore nullo) e opposti ($-v$), $\cdot$ é associativa, distribuitiva rispetto alla somma e con elemento neutro.\\\\
Per ogni campo $K$, $K^n$ é uno spazio vettoriale su $K$.\\
$K^n = \{(x_1, x_2, ..., x_n), x_i \in K, \forall i=1, ..., n\} \\
v = (x_1, x_2, ..., x_n), u = (y_1, y_2, ..., y_n) \\
v + u = (x_1 + y_1, x_2 + y_2, ..., x_n + y_n) \\
av = (ax_1, ax_2, ..., ax_n), a \in K
$

\end{document}