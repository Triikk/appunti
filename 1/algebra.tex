\documentclass{article}
\usepackage{graphicx} % Required for inserting images
\usepackage{amsmath}
\usepackage{mathtools}
\usepackage{amsfonts} % \mathbb
% \usepackage{centernot}

\title{
	Appunti di Algebra \\
	\large A.A. 2022/2023
}
\date{}

\begin{document}

\maketitle

\section{Introduzione}

\subsection{Relazioni}
Una \textbf{relazione} é un sottoinsieme del prodotto cartesiano di due o piú insiemi.\\
Una relazione su $A$ é un sottoinsieme di $A \times A$. \\\\
$a_{1}$ é in relazione con $a_{2}$ e si scrive $a_{1} R a_{2}$. \\
Def. Una relazione é di \textbf{equivalenza} se rispetta le seguenti proprietá:
\begin{itemize}
	\item[] Riflessiva: $a R a$ $\forall a \in A$ (ogni elemento é in relazione con se stesso)
	\item[] Simmetrica: $a_{1} R a_{2} \implies a_{2} R a_{1}$ $\forall a_{1}, a_{2} \in A$
	\item[] Transitiva: $a_1 R a_2 \wedge a_2 R a_3 \implies a_1 R a_3$
\end{itemize}

\subsection{Funzioni/Applicazioni}
$f : X \rightarrow Y$\\\\
$f$ iniettiva: $\forall x_1, x_2 \in X, f(x_1) = f(x_2) \implies x_1=x_2$ \\
$f$ suriettiva: $\forall y \in Y, \exists x \in A : y = f(x)$ \\
$f$ biettiva: $\forall y \in Y, \exists! x \in A : y = f(x)$

\subsection{Insiemi numerici}
L'insieme dei numeri razionali $\mathbb{Q}$ introduce gli inversi del prodotto (es. $3 \rightarrow \frac{1}{3}$). \\
L'insieme dei numeri reali $\mathbb{R}$ introduce limiti, radici e altri valori. \\
L'insieme dei numeri complessi $\mathbb{C}$ introduce le radici di indice pari di numeri negativi tramite l'unitá immaginaria $i$ e i suoi multipli. Un numero complesso é esprimibile in forma polare come $a+ib$, con $a,b \in R$.

\subsection{Campi}
$(K, +, \cdot)$ é un campo se:
\begin{itemize}
	\item[] $+, \cdot$ sono associative ($a+(b+c) = (a+b)+c$), commutative ($a+b=b+a$) e distributive ($a(b+c)=ab+ac$)
	\item[] esistono elementi \textbf{neutri} ($0$ per la somma ($a+0=a$), $1$ per il prodotto ($a \cdot 1 = a$)) e \textbf{opposti} ($-a$ per la somma ($a-a=0$), $x^{-1}$ per il prodotto ($x \cdot x^{-1} = 1$), che restituscono il valore neutro
\end{itemize}
Alcuni insiemi campi sono $\mathbb{Q}, \mathbb{R}, \mathbb{C}$.

\subsubsection{Campi finiti}
Dato un numero intero $n \ge 0$, definiamo su $\mathbb{Z}$ la relazione di equivalenza
$$a \equiv b (n) \iff \exists k \in Z : a - b = k \cdot n$$
essa rispetta tutte e 3 le proprietá elencate sopra. \\\\
Definiamo $[b] = \{a \in Z : a \equiv b (n)\}$ e $Z_n = \{[0], [1], ..., [n-1]\}$. \\
Es. in $Z_2 = \{[0], [1]\}$, $[0]$ sono i numeri pari, $[1]$ quelli dispari. \\\\
Definiamo su $Z_n$ le operazioni:\\
$$[a] + [b] = [a+b], [a] \cdot [b] = [a \cdot b]$$ \\
Es. Possiamo scrivere, con la notazione dei campi finiti, il prodotto tra numeri interi: \\
Dato $Z_2$: $[0] \cdot [0] = [0], [0] \cdot [1] = [0 \cdot 1] = [0], [1] \cdot [1] = [1 \cdot 1] = [1]$. \\\\
$Z_n$ é un campo $\iff n$ é \textbf{primo}. Se $n$ non é primo, non esisterá l'inverso di un fattore di $n$, ovvero non esisterá nessuna classe di elementi che se moltiplicata con la classe del fattore restituisca classe 1.

\subsection{Spazi vettoriali}
Uno \textbf{spazio vettoriale} definito su un campo $K$ é un insieme $V$ con due operazioni:
\begin{itemize}
	\item[] $+ : V \times V \rightarrow V$ ($v_1, v_2) \rightarrow v_1 + v_2$)
	\item[] $\cdot : K \times V \rightarrow V$ ($a, v \rightarrow av$)
\end{itemize}
che verificano le seguenti proprietá: $+$ é commutativa, associativa, con elem. neutri (vettore nullo) e opposti ($-v$), $\cdot$ é associativa, distribuitiva rispetto alla somma e con elemento neutro.\\\\
Per ogni campo $K$, $K^n$ é uno spazio vettoriale su $K$.\\
$K^n = \{(x_1, x_2, ..., x_n), x_i \in K, \forall i=1, ..., n\} \\
v = (x_1, x_2, ..., x_n), u = (y_1, y_2, ..., y_n) \\
v + u = (x_1 + y_1, x_2 + y_2, ..., x_n + y_n) \\
av = (ax_1, ax_2, ..., ax_n), a \in K
$

\subsection{Sottospazi vettoriali}
Un sottoinsieme non vuoto (contenente almeno il vettore nullo) $U \subseteq V$ (spazio vettoriale su $K$) é un \textbf{sottospazio vettoriale} (SSV) di V se é \textbf{chiuso} rispetto alle sue operazioni, cioé:
\begin{itemize}
	\item $\forall v_1, v_2 \in U \rightarrow v_1 + v_2 \in U$
	\item $\forall v_1 \in U, a \in K, a \cdot v_1 \in U$
\end{itemize}
Esempio: $V = R^2$ spazio vettoriale su R, $U = \{(x,y) \in R^2 : y = 2x\}$. É un SSV?\\
Se $v_1, v_2 \in U: v_1 = (x_1, y_1) \rightarrow y_1 = 2x_1, v_2 = (x_2, y_2) \rightarrow y_2 = 2x_2$\\
$v_1 + v_2 = (x_1 + x_2, 2x_1 + 2x_2) \rightarrow (x_1 + x_2, 2(x_1 + x_2)) \implies v_1 + v_2 \in U$. Inoltre, $\forall a \in R, a \cdot v_1 = a(x_1, 2ax_1) \implies a \cdot v_1 \in U$. Quindi, $U$ é un SSV di $V$.\\
Graficamente, significa che la somma di qualsiasi coppia di vettori presenti sulla retta $y=2x$ é un vettore sempre giacente su questa retta, cosí come il prodotto di qualsiasi vettore giacente sulla retta per un qualsiasi scalare é un vettore sempre giacente su questa retta.\\\\
$U$ é un SSV di $V \iff \forall u_1, u_2 \in U, \forall a_1, a_2 \in K$.\\
Dimostrazione:
\begin{itemize}
	\item[$\Rightarrow$]: se $U$ é un SSV di V e $u_1, u_2 \ in U \implies a_1 u_1, a_2 u_2 \in U \implies a_1 u_1 + a_2 u_2$
	\item[$\Leftarrow$]: se $a_1 u_1 + a_2 u_2 \in U \forall a_1, a_2 \in K$, in particolare: prendendo $a_1 = 1, a_2 = 1, u_1 + u_2 \in U$, prendendo $a_1$ qualsiasi e $a_2 = 0, a_1 u_1 \in U$
\end{itemize}

\subsection{Combinazione lineare}
Dati $v_1, v_2, \dots, v_n \in V$, diciamo che $v \in V$ é una \textbf{combinazione lineare} di $v_1, v_2, \dots, v_n$ se $\exists a_1, a_2, \dots, a_n \in K : v = a_1 v_1 + a_2 v_2 + \dots + a_n v_n$, quindi se $v$ é esprimibile come la somma di tutti i vettori di $V$ moltiplicati per un corrispettivo scalare.\\
Per quanto osservato sopra, $U$ é un SSV $\iff$ contiene \textit{tutte le combinazioni lineari} di \textit{tutti i suoi elementi}. Non esiste una combinazione lineare $u$ degli elementi di $U$ che non sia $\in U$.\\\\
Esempio: $V = R^2, v_1 = (1,0), v_2 = (0,1), v_3 = (3, -2)$. $v_3 = 3v_1 -2v_2$ quindi $v_3$ é combinazione lineare di $v_1, v_2$. Altro esempio: $u_1 = (2,0), u_2 = (-1, 0), u_3 = (3, -2)$. $u$ in questo caso non é combinazione lineare di $u_1, u_2$ perché non é possibile ottenere la seconda coordinata -2 essendo 0 in entrambi.

\subsection{Span}
Un SSV $U$ di $V$ é \textbf{generato} da $\{v_1, \dots, v_n\}$ se ogni elemento $u \in U$ é combinazione lineare di $v_1, \dots, v_n$, cioé se $\forall u \ in U, \exists a_1, \dots, a_n \in K : u = a_1 v_1 + \dots + a_n v_n$. $U$ é lo \textbf{span} di $\{v_1, \dots, v_n\}$ ed é scritto $U = \langle v_1, \dots, v_n \rangle$. Si noti che l'insieme contiene infiniti elementi, siccome infinite sono le combinazioni lineari ottenibili ($a_1, \dots, a_n \in R$).\\\\
Esempio: $
V = R^4 = \{(x,y,z,w), x,y,z,w \in R\}, \\
v_1 = (2,0,0,0), v_2 = (0,1,-1,0)
$. Il sottospazio generato da $v_1, v_2$ é $\langle v_1, v_2 \rangle = \{a_1 v_1 + a_2 v_2, a_1, a_2 \in R\} = (2a_1,0,0,0) + (0,a_2,-a_2,0) = (2a_1,a_2,-a_2,0) = \{(x,y,z,w) \in R^4 : y + z = 0, w = 0\}$.\\
Altro esempio: $V = R[x], \langle x^2, x, 1 \rangle = \{ax^2+bx+c, a,b,c \in R\} = \{$ tutti i polinomi di grado $\leq 2\}$ (con $a=0$ il grado é $< 2$).

\subsection{Indipendenza lineare}
Un insieme di vettori $\{v_1, \dots, v_n\}$ é \textbf{linearmente indipendente} se \textit{nessun vettore} é la combinazione lineare degli altri vettori dell'insieme, ovvero se l'\textit{unica combinazione lineare di} $v_1, \dots, v_n$ che restituisce il vettore nullo é quella con tutti i coefficienti $a_1, \dots, a_n \in K = 0$. Questo perché se un vettore é combinazione lineare di un insieme di vettori (es. $v_3 = 2v_1 + 4v_2$), basta dare i giusti coefficienti ($a_1=2, a_2=4, a3=-1)$ per fare in modo che si annullino e mettere i coefficienti degli altri vettori a 0.\\\\
Un insieme di vettori é \textbf{linearmente dipendente} se non é linearmente indipendente. Non é peró detto che ogni vettore apparentenente a un insieme linearmente dipendente sia combinazione lineare di altri (es. $v_1 = (1,0), v_2 = (2,0), v_3 = (0,1) \rightarrow v_2=2v_1$ ma $v_3$ non é combinazione lineare di $v_1, v_2$), é sufficiente che una coppia di vettori sia ricavabile l'una dall'altra per rendere tutto l'insieme linearmente dipendente.

\subsubsection{Equivalenza definizioni di indipendenza lineare}
\begin{enumerate}
	\item Nessun vettore tra $v_1, \dots, v_n$ é combinazione lineare degli altri
	\item Se $a_1v_1 + \dots + a_nv_n = 0 \implies a_1=0, \dots,a_n=0$
\end{enumerate}
Se la 1 é falsa, $\exists v_i$ (supponiamo per semplicitá sia $v_1$) che é combinazione lineare degli altri, quindi $v_1 = a_2v_2 + \dots + a_nv_n \iff v_1 - a_2v_2 - \dots - a_nv_n = 0 \implies$ la 2 é anch'essa falsa perché i coefficienti non sono per forza tutti 0 (sicuramente $a_1=1$).\\
Se la 2 é falsa significa che $\exists a_1, \dots, a_n$ con almeno un $a_i \ne 0 : a_1v_1 + \dots + a_nv_n = 0$, allora $\displaystyle v_i = \frac{a_1}{a_i}v_1 + \dots + \frac{a_n}{a_i}v_n \implies$ la 1 é anch'essa falsa siccome $v_i$ é combinazione lineare degli altri.

\subsection{Base}
Sia $V$ uno spazio vettoriale su $K$, $\{v_1, \dots, v_n\}$ é una \textbf{base} di $V$ se $\{v_1, \dots, v_n\}$ é \textbf{indipendente} e \textbf{genera V}.\\
Ad esempio, per $V=R^2, \{(1,0),(0,1)\}$ é indipendente e genera $R^2$, quindi é una base, mentre $\{(1,0), (0,1), (1,1)\}$ genera $R^2$ ma é lineramente dipendente, quindi non é una base.

\end{document}